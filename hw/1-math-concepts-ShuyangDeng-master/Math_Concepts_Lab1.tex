\documentclass[11pt,]{article}
\usepackage{lmodern}
\usepackage{amssymb,amsmath}
\usepackage{ifxetex,ifluatex}
\usepackage{fixltx2e} % provides \textsubscript
\ifnum 0\ifxetex 1\fi\ifluatex 1\fi=0 % if pdftex
  \usepackage[T1]{fontenc}
  \usepackage[utf8]{inputenc}
\else % if luatex or xelatex
  \ifxetex
    \usepackage{mathspec}
  \else
    \usepackage{fontspec}
  \fi
  \defaultfontfeatures{Ligatures=TeX,Scale=MatchLowercase}
\fi
% use upquote if available, for straight quotes in verbatim environments
\IfFileExists{upquote.sty}{\usepackage{upquote}}{}
% use microtype if available
\IfFileExists{microtype.sty}{%
\usepackage{microtype}
\UseMicrotypeSet[protrusion]{basicmath} % disable protrusion for tt fonts
}{}
\usepackage[margin=1in]{geometry}
\usepackage{hyperref}
\hypersetup{unicode=true,
            pdftitle={Math Concepts},
            pdfauthor={Anna Yeaton},
            pdfborder={0 0 0},
            breaklinks=true}
\urlstyle{same}  % don't use monospace font for urls
\usepackage{color}
\usepackage{fancyvrb}
\newcommand{\VerbBar}{|}
\newcommand{\VERB}{\Verb[commandchars=\\\{\}]}
\DefineVerbatimEnvironment{Highlighting}{Verbatim}{commandchars=\\\{\}}
% Add ',fontsize=\small' for more characters per line
\usepackage{framed}
\definecolor{shadecolor}{RGB}{248,248,248}
\newenvironment{Shaded}{\begin{snugshade}}{\end{snugshade}}
\newcommand{\KeywordTok}[1]{\textcolor[rgb]{0.13,0.29,0.53}{\textbf{#1}}}
\newcommand{\DataTypeTok}[1]{\textcolor[rgb]{0.13,0.29,0.53}{#1}}
\newcommand{\DecValTok}[1]{\textcolor[rgb]{0.00,0.00,0.81}{#1}}
\newcommand{\BaseNTok}[1]{\textcolor[rgb]{0.00,0.00,0.81}{#1}}
\newcommand{\FloatTok}[1]{\textcolor[rgb]{0.00,0.00,0.81}{#1}}
\newcommand{\ConstantTok}[1]{\textcolor[rgb]{0.00,0.00,0.00}{#1}}
\newcommand{\CharTok}[1]{\textcolor[rgb]{0.31,0.60,0.02}{#1}}
\newcommand{\SpecialCharTok}[1]{\textcolor[rgb]{0.00,0.00,0.00}{#1}}
\newcommand{\StringTok}[1]{\textcolor[rgb]{0.31,0.60,0.02}{#1}}
\newcommand{\VerbatimStringTok}[1]{\textcolor[rgb]{0.31,0.60,0.02}{#1}}
\newcommand{\SpecialStringTok}[1]{\textcolor[rgb]{0.31,0.60,0.02}{#1}}
\newcommand{\ImportTok}[1]{#1}
\newcommand{\CommentTok}[1]{\textcolor[rgb]{0.56,0.35,0.01}{\textit{#1}}}
\newcommand{\DocumentationTok}[1]{\textcolor[rgb]{0.56,0.35,0.01}{\textbf{\textit{#1}}}}
\newcommand{\AnnotationTok}[1]{\textcolor[rgb]{0.56,0.35,0.01}{\textbf{\textit{#1}}}}
\newcommand{\CommentVarTok}[1]{\textcolor[rgb]{0.56,0.35,0.01}{\textbf{\textit{#1}}}}
\newcommand{\OtherTok}[1]{\textcolor[rgb]{0.56,0.35,0.01}{#1}}
\newcommand{\FunctionTok}[1]{\textcolor[rgb]{0.00,0.00,0.00}{#1}}
\newcommand{\VariableTok}[1]{\textcolor[rgb]{0.00,0.00,0.00}{#1}}
\newcommand{\ControlFlowTok}[1]{\textcolor[rgb]{0.13,0.29,0.53}{\textbf{#1}}}
\newcommand{\OperatorTok}[1]{\textcolor[rgb]{0.81,0.36,0.00}{\textbf{#1}}}
\newcommand{\BuiltInTok}[1]{#1}
\newcommand{\ExtensionTok}[1]{#1}
\newcommand{\PreprocessorTok}[1]{\textcolor[rgb]{0.56,0.35,0.01}{\textit{#1}}}
\newcommand{\AttributeTok}[1]{\textcolor[rgb]{0.77,0.63,0.00}{#1}}
\newcommand{\RegionMarkerTok}[1]{#1}
\newcommand{\InformationTok}[1]{\textcolor[rgb]{0.56,0.35,0.01}{\textbf{\textit{#1}}}}
\newcommand{\WarningTok}[1]{\textcolor[rgb]{0.56,0.35,0.01}{\textbf{\textit{#1}}}}
\newcommand{\AlertTok}[1]{\textcolor[rgb]{0.94,0.16,0.16}{#1}}
\newcommand{\ErrorTok}[1]{\textcolor[rgb]{0.64,0.00,0.00}{\textbf{#1}}}
\newcommand{\NormalTok}[1]{#1}
\usepackage{graphicx,grffile}
\makeatletter
\def\maxwidth{\ifdim\Gin@nat@width>\linewidth\linewidth\else\Gin@nat@width\fi}
\def\maxheight{\ifdim\Gin@nat@height>\textheight\textheight\else\Gin@nat@height\fi}
\makeatother
% Scale images if necessary, so that they will not overflow the page
% margins by default, and it is still possible to overwrite the defaults
% using explicit options in \includegraphics[width, height, ...]{}
\setkeys{Gin}{width=\maxwidth,height=\maxheight,keepaspectratio}
\IfFileExists{parskip.sty}{%
\usepackage{parskip}
}{% else
\setlength{\parindent}{0pt}
\setlength{\parskip}{6pt plus 2pt minus 1pt}
}
\setlength{\emergencystretch}{3em}  % prevent overfull lines
\providecommand{\tightlist}{%
  \setlength{\itemsep}{0pt}\setlength{\parskip}{0pt}}
\setcounter{secnumdepth}{0}
% Redefines (sub)paragraphs to behave more like sections
\ifx\paragraph\undefined\else
\let\oldparagraph\paragraph
\renewcommand{\paragraph}[1]{\oldparagraph{#1}\mbox{}}
\fi
\ifx\subparagraph\undefined\else
\let\oldsubparagraph\subparagraph
\renewcommand{\subparagraph}[1]{\oldsubparagraph{#1}\mbox{}}
\fi

%%% Use protect on footnotes to avoid problems with footnotes in titles
\let\rmarkdownfootnote\footnote%
\def\footnote{\protect\rmarkdownfootnote}

%%% Change title format to be more compact
\usepackage{titling}

% Create subtitle command for use in maketitle
\newcommand{\subtitle}[1]{
  \posttitle{
    \begin{center}\large#1\end{center}
    }
}

\setlength{\droptitle}{-2em}

  \title{Math Concepts}
    \pretitle{\vspace{\droptitle}\centering\huge}
  \posttitle{\par}
    \author{Anna Yeaton}
    \preauthor{\centering\large\emph}
  \postauthor{\par}
      \predate{\centering\large\emph}
  \postdate{\par}
    \date{Fall 2018}


\begin{document}
\maketitle

\section{Math Concepts}\label{math-concepts}

In this lab, we will go over math concepts that serve as a foundation
for the unsupervised learning class. We will go over standard deviation,
variance, covariance, eigen values, and eigen vectors.

\subsection{Distributions}\label{distributions}

\begin{Shaded}
\begin{Highlighting}[]
\CommentTok{#Gaussian(normal) distribution}
\NormalTok{normal_dist <-}\StringTok{ }\KeywordTok{rnorm}\NormalTok{(}\DataTypeTok{n =} \DecValTok{100}\NormalTok{, }\DataTypeTok{mean =} \DecValTok{10}\NormalTok{, }\DataTypeTok{sd =} \DecValTok{5}\NormalTok{)}
\KeywordTok{plot}\NormalTok{(normal_dist, }\DataTypeTok{ylab=}\StringTok{"Y"}\NormalTok{, }\DataTypeTok{xlab =} \StringTok{"X"}\NormalTok{)}
\end{Highlighting}
\end{Shaded}

\includegraphics{Math_Concepts_Lab1_files/figure-latex/unnamed-chunk-1-1.pdf}

\begin{Shaded}
\begin{Highlighting}[]
\KeywordTok{hist}\NormalTok{(normal_dist)}
\end{Highlighting}
\end{Shaded}

\includegraphics{Math_Concepts_Lab1_files/figure-latex/unnamed-chunk-1-2.pdf}

\begin{Shaded}
\begin{Highlighting}[]
\CommentTok{#exponential distribution}
\NormalTok{exponential_dist <-}\StringTok{ }\KeywordTok{rexp}\NormalTok{(}\DataTypeTok{n=}\DecValTok{100}\NormalTok{, }\DataTypeTok{rate =}\DecValTok{1}\NormalTok{)}
\KeywordTok{plot}\NormalTok{(exponential_dist, }\DataTypeTok{ylab=}\StringTok{"Y"}\NormalTok{, }\DataTypeTok{xlab =} \StringTok{"X"}\NormalTok{)}
\end{Highlighting}
\end{Shaded}

\includegraphics{Math_Concepts_Lab1_files/figure-latex/unnamed-chunk-1-3.pdf}

\begin{Shaded}
\begin{Highlighting}[]
\KeywordTok{hist}\NormalTok{(exponential_dist)}
\end{Highlighting}
\end{Shaded}

\includegraphics{Math_Concepts_Lab1_files/figure-latex/unnamed-chunk-1-4.pdf}

\begin{Shaded}
\begin{Highlighting}[]
\CommentTok{#lognormal distribution}
\NormalTok{lognormal_distribution <-}\StringTok{ }\KeywordTok{rlnorm}\NormalTok{(}\DecValTok{100}\NormalTok{, }\DataTypeTok{meanlog =} \DecValTok{10}\NormalTok{, }\DataTypeTok{sdlog =} \DecValTok{5}\NormalTok{)}
\KeywordTok{plot}\NormalTok{(lognormal_distribution,  }\DataTypeTok{ylab=}\StringTok{"Y"}\NormalTok{, }\DataTypeTok{xlab =} \StringTok{"X"}\NormalTok{)}
\end{Highlighting}
\end{Shaded}

\includegraphics{Math_Concepts_Lab1_files/figure-latex/unnamed-chunk-1-5.pdf}

\begin{Shaded}
\begin{Highlighting}[]
\KeywordTok{hist}\NormalTok{(lognormal_distribution)}
\end{Highlighting}
\end{Shaded}

\includegraphics{Math_Concepts_Lab1_files/figure-latex/unnamed-chunk-1-6.pdf}

\begin{Shaded}
\begin{Highlighting}[]
\KeywordTok{hist}\NormalTok{(}\KeywordTok{log}\NormalTok{(lognormal_distribution))}
\end{Highlighting}
\end{Shaded}

\includegraphics{Math_Concepts_Lab1_files/figure-latex/unnamed-chunk-1-7.pdf}

\subsection{Variance}\label{variance}

\[variance=\frac{1}{N}\sum_{i=1}^{n}(x_i - \bar{x})(x_i - \bar{x})\]
Variance is a measurememt of the spread of data.

\begin{Shaded}
\begin{Highlighting}[]
\CommentTok{#the spread of this set of values }
\NormalTok{spread_ex1 <-}\StringTok{ }\KeywordTok{c}\NormalTok{(}\DecValTok{1}\NormalTok{,}\DecValTok{2}\NormalTok{,}\DecValTok{1}\NormalTok{,}\DecValTok{2}\NormalTok{,}\DecValTok{1}\NormalTok{,}\DecValTok{2}\NormalTok{,}\DecValTok{1}\NormalTok{,}\DecValTok{2}\NormalTok{,}\DecValTok{1}\NormalTok{,}\DecValTok{2}\NormalTok{)}

\CommentTok{#is much different from the spread of this set of values}
\NormalTok{spread_ex2 <-}\StringTok{ }\KeywordTok{c}\NormalTok{(}\DecValTok{2}\NormalTok{,}\DecValTok{5}\NormalTok{,}\DecValTok{8}\NormalTok{,}\DecValTok{10}\NormalTok{,}\DecValTok{1}\NormalTok{,}\DecValTok{35}\NormalTok{,}\DecValTok{67}\NormalTok{,}\DecValTok{32}\NormalTok{,}\DecValTok{89}\NormalTok{,}\DecValTok{100}\NormalTok{)}
\end{Highlighting}
\end{Shaded}

Calculate the variance of spread\_ex1 and spread\_ex2

\begin{Shaded}
\begin{Highlighting}[]
\KeywordTok{var}\NormalTok{(spread_ex1)}
\end{Highlighting}
\end{Shaded}

\begin{verbatim}
## [1] 0.2777778
\end{verbatim}

\begin{Shaded}
\begin{Highlighting}[]
\KeywordTok{var}\NormalTok{(spread_ex2)}
\end{Highlighting}
\end{Shaded}

\begin{verbatim}
## [1] 1408.1
\end{verbatim}

\subsection{Standard deviation}\label{standard-deviation}

\[stdv=\sqrt{\frac{1}{N}\sum_{i=1}^{n}(x_i - \bar{x})(x_i - \bar{x})}\]

Standard deviation is also a measurememt of the spread of data. It is
the square root of the variance.

Calculate the standard deviations of spread\_ex1 and spread\_ex2

\begin{Shaded}
\begin{Highlighting}[]
\KeywordTok{sd}\NormalTok{(spread_ex1)}
\end{Highlighting}
\end{Shaded}

\begin{verbatim}
## [1] 0.5270463
\end{verbatim}

\begin{Shaded}
\begin{Highlighting}[]
\KeywordTok{sd}\NormalTok{(spread_ex2)}
\end{Highlighting}
\end{Shaded}

\begin{verbatim}
## [1] 37.52466
\end{verbatim}

\subsection{Covariance}\label{covariance}

Covariance is a measure of the variance of two random variables with
respect to each other. In other words, covariance is a measure of linear
dependence between two random variables.

\[covariance=\frac{1}{N}\sum_{i=1}^{n}(x_i - \bar{x})(y_i - \bar{y})\]

\begin{Shaded}
\begin{Highlighting}[]
\NormalTok{cov_ex1 <-}\StringTok{ }\KeywordTok{data.frame}\NormalTok{(spread_ex1, spread_ex2)}
\end{Highlighting}
\end{Shaded}

Calculate the covariance of cov\_ex1 between spread\_ex1 and spread\_ex2

\begin{Shaded}
\begin{Highlighting}[]
\KeywordTok{cov}\NormalTok{(cov_ex1}\OperatorTok{$}\NormalTok{spread_ex1, cov_ex1}\OperatorTok{$}\NormalTok{spread_ex2)}
\end{Highlighting}
\end{Shaded}

\begin{verbatim}
## [1] 0.8333333
\end{verbatim}

Calculate the covariance matrix of cov\_ex1. Did you see any of these
values before?

\begin{Shaded}
\begin{Highlighting}[]
\KeywordTok{cov}\NormalTok{(cov_ex1)}
\end{Highlighting}
\end{Shaded}

\begin{verbatim}
##            spread_ex1   spread_ex2
## spread_ex1  0.2777778    0.8333333
## spread_ex2  0.8333333 1408.1000000
\end{verbatim}

\subsection{Eigen vectors and Eigen
values}\label{eigen-vectors-and-eigen-values}

Eigen vectors and Eigen values come in pairs. An eigen vector is a
direction, and the corresponding eigen value tells how much the data
varies in that direction. You can only calculate eigen vectors and
values for n x n matrices. For an n x n matrix, there will be n eigen
values and eigen vectors. Eigen vectors are perpendicular to each other.
The eigen vector with the highest eigen value is the principal
component.

So how do you find an eigen vector? Why is it one direction and not
another? The eigen vector is special in that when a linear
transformation is applied to it, the direction of the vector does not
change.

You can project the data onto an eigen vector space instead of on the X
and Y axis. This is the visualization for PCA.

\subsection{Matrix operations}\label{matrix-operations}

\url{http://home.cc.umanitoba.ca/~thomas/Courses/MatrixMultPractice.pdf}


\end{document}
